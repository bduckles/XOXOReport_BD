% Options for packages loaded elsewhere
\PassOptionsToPackage{unicode}{hyperref}
\PassOptionsToPackage{hyphens}{url}
%
\documentclass[
]{book}
\usepackage{amsmath,amssymb}
\usepackage{iftex}
\ifPDFTeX
  \usepackage[T1]{fontenc}
  \usepackage[utf8]{inputenc}
  \usepackage{textcomp} % provide euro and other symbols
\else % if luatex or xetex
  \usepackage{unicode-math} % this also loads fontspec
  \defaultfontfeatures{Scale=MatchLowercase}
  \defaultfontfeatures[\rmfamily]{Ligatures=TeX,Scale=1}
\fi
\usepackage{lmodern}
\ifPDFTeX\else
  % xetex/luatex font selection
\fi
% Use upquote if available, for straight quotes in verbatim environments
\IfFileExists{upquote.sty}{\usepackage{upquote}}{}
\IfFileExists{microtype.sty}{% use microtype if available
  \usepackage[]{microtype}
  \UseMicrotypeSet[protrusion]{basicmath} % disable protrusion for tt fonts
}{}
\makeatletter
\@ifundefined{KOMAClassName}{% if non-KOMA class
  \IfFileExists{parskip.sty}{%
    \usepackage{parskip}
  }{% else
    \setlength{\parindent}{0pt}
    \setlength{\parskip}{6pt plus 2pt minus 1pt}}
}{% if KOMA class
  \KOMAoptions{parskip=half}}
\makeatother
\usepackage{xcolor}
\usepackage{longtable,booktabs,array}
\usepackage{calc} % for calculating minipage widths
% Correct order of tables after \paragraph or \subparagraph
\usepackage{etoolbox}
\makeatletter
\patchcmd\longtable{\par}{\if@noskipsec\mbox{}\fi\par}{}{}
\makeatother
% Allow footnotes in longtable head/foot
\IfFileExists{footnotehyper.sty}{\usepackage{footnotehyper}}{\usepackage{footnote}}
\makesavenoteenv{longtable}
\usepackage{graphicx}
\makeatletter
\def\maxwidth{\ifdim\Gin@nat@width>\linewidth\linewidth\else\Gin@nat@width\fi}
\def\maxheight{\ifdim\Gin@nat@height>\textheight\textheight\else\Gin@nat@height\fi}
\makeatother
% Scale images if necessary, so that they will not overflow the page
% margins by default, and it is still possible to overwrite the defaults
% using explicit options in \includegraphics[width, height, ...]{}
\setkeys{Gin}{width=\maxwidth,height=\maxheight,keepaspectratio}
% Set default figure placement to htbp
\makeatletter
\def\fps@figure{htbp}
\makeatother
\setlength{\emergencystretch}{3em} % prevent overfull lines
\providecommand{\tightlist}{%
  \setlength{\itemsep}{0pt}\setlength{\parskip}{0pt}}
\setcounter{secnumdepth}{5}
\usepackage{booktabs}
\ifLuaTeX
  \usepackage{selnolig}  % disable illegal ligatures
\fi
\usepackage[]{natbib}
\bibliographystyle{plainnat}
\usepackage{bookmark}
\IfFileExists{xurl.sty}{\usepackage{xurl}}{} % add URL line breaks if available
\urlstyle{same}
\hypersetup{
  pdftitle={XOXO Community Survey 2024},
  pdfauthor={Beth M. Duckles},
  hidelinks,
  pdfcreator={LaTeX via pandoc}}

\title{XOXO Community Survey 2024}
\author{Beth M. Duckles}
\date{2024-10-20}

\begin{document}
\maketitle

{
\setcounter{tocdepth}{1}
\tableofcontents
}
\chapter{XOXO Community Survey 2024}\label{xoxo-community-survey-2024}

Welcome to the XOXO Community Survey 2024 report. This report covers various aspects of community engagement and feedback collected during the survey.

\section{Table of Contents}\label{table-of-contents}

\begin{itemize}
\tightlist
\item
  \href{01-intro.html}{Introduction}
\item
  \href{02-ValuesCommty.html}{Community Values}
\item
  \href{03-AppreciateCmmty.html}{Appreciation For the Community}
\item
  \href{04-Concerns.html}{Concerns in the Community}
\item
  \href{09-Impact.html}{Impact of the Slack}
\end{itemize}

\chapter{Introduction}\label{introduction}

\section{What this is}\label{what-this-is}

This is a short, vibe check community survey of the XOXO community. In early October 2024, Andy McMillan and Andy Baio (sometimes collectively referred to as ``the Andys'') announced that because the festival had ended this year, they would be closing down the XOXO Slack community that has been going since the beginning of the festival.

There was a lot of discussion in the slack and some folks suggested we might want to do a survey to discuss next directions. In consultation with Pete Modica-Soloway, Josh Millard and Nora Ryan, I (Beth Duckles) put together this survey to start the conversation.

We did a short one, mostly written survey with qualitative feedback to the community offering about a week to respond. There were 154 responses to the survey.

This report is meant to summarize the findings and to pull out key themes so as to start the conversation.

\section{Lower your expectations}\label{lower-your-expectations}

I have done a quick organization of the comments for each question into themes as a way to share the information that came from the survey. In each theme, I add a bunch of quotes that describe what I'm seeing.

The qualitative coding/tagging (meaning grouping responses into themes) is rough, imperfect and it was done very quickly. I wanted the community have something fast more than I wanted to perfect it.

That means there might be a theme you see that I didn't create, or something that fits with multiple themes. You might disagree with how I named the groupings, or you might think a quote in a theme is wrong altogether.

All of this is great.

If you think something is was placed in the wrong spot, or that I missed a theme or that it's named wrong, you are probably right. You're invited to let us know, point out mistakes or to do another pass at it on your own or read the raw data in the dataset if you'd prefer that.

The goal for this for paper was to create an easier way to start a community discussion, not a final ``in stone'' explanation of what everyone was saying. By reflecting what I saw/read in the data, I'm starting the process of summarizing to get our conversation going.

Please, don't worry about my feelings and go ahead and converse.

It's also LONG, so I understand that it might feel overwhelming. I made it long because I feel strongly that listening to the many voices of other people involves letting their words be at the center. So I include a lot (a lot) of quotes.

This may be overwhelming. I am sorry about that. I've done my best to summarize so you don't have to read it all unless you want to. I encourage you to skip sections if you don't want to read something, or just go straight to the comments for the Andy's and commence weeping. Or not. You do you.

There are likely a lot of typos. If you feel strongly about this and want to contribute, you are warmly invited to copy edit what I've written. However, in this kind of research, I don't edit what people write in their response to the survey. This means that if there are typos in what people wrote, there are typos in how it is shown here. This is not meant to shame anyone, but rather to preserve what people say and how they say it. I don't presume I know better how to say what someone else is saying than they do.

\section{Gift}\label{gift}

This work is a gift freely given to a community that has given me so so much.

It is also something I've been able to do because I am between projects and I have free time right now as a freelance researcher. Which means, I am looking for work. This is a small section of the kind of things I can do research-wise. Just think of how much more I could do if I was paid!

If you want to talk to me about doing research projects for hire or want to drop a few bucks my way, I'm always up for it - find me on the slack at bduckles or \href{mailto:beth@duckles.com}{\nolinkurl{beth@duckles.com}}.

\chapter{Community Values}\label{community-values}

The question we asked was: ``If you were to briefly summarize the core values of the XOXO Slack Community as it exists now, what would they be?

Many folks responded with a string of words separated by commas, so I pulled those apart, counted them and collapsed where appropriate (e.g.~creative, creativity and creations became creativity). Others wrote a phrase or a sentence and those are interspersed in the discussion where they added to the themes.

The most popular words folks used were: (approximate number in parentheses):

\begin{itemize}
\tightlist
\item
  Empathy (27)
\item
  Kindness (26)
\item
  Creative (25)
\item
  Openness (14)
\item
  Community (15)
\item
  Curiosity (13)
\item
  Inclusivity (12)
\item
  Thoughtful (11)
\item
  Support (9)
\item
  Compassionate (9)
\item
  Fun (7)
\item
  Caring (6)
\item
  Respect (6)
\item
  Enthusiastic (5)
\item
  Helpful (5)
\item
  Generosity (4)
\item
  Friendly (3)
\item
  Sharing (3)
\end{itemize}

I grouped the words together in themes that I came up with based on what folks wrote/said. I added quotes to match the groupings. All groupings are open to discussion/interpretation and clearly there is a lot of overlap.

\section{Community}\label{community}

Quite a few people used words related to community such as: support, sharing, acceptance, generosity, mutual support, supportive, camaraderie, mutual aid, togetherness, care for each other, cooperation, equity, giving voice to the marginalized, giving others the benefit of the doubt, lifting each other up, mutual respect, togetherness, authenticity, tolerance, trust, utopianism and validation.

Quotes on community:

\begin{itemize}
\tightlist
\item
  thoughtful community based on lifting up and informing each other
\item
  Inclusive community
\item
  Unique set of people with a lot of things in common that supports each other and shares good stuff
\end{itemize}

\section{Creative/Weird/Fun}\label{creativeweirdfun}

Creative was the most used word, and it seemed to also encompass the idea of artistry and a sense of fun. Folks used words like: enthusiastic, humor, entertaining, exciting, making weird stuff for fun, playful, sharing fun things, sharing neat projects and weirdness.

Some quotes around this theme:

\begin{itemize}
\tightlist
\item
  Give creative, smart, good-hearted people a safe way to express themselves, connect with others, learn to share themselves with the world, and live fulfilling sustainable creative lives.
\item
  A welcoming community for the best weirdos the internet has to offer
\item
  A community of people who build good internet-related things (loosely defined). People who are humane and humorous. People who want to share things that they find joyful.
\item
  Creative, curious, KIND people. Safe space to explore ideas and thoughts about life and creative work!
\item
  take joy in the things around you; celebrate things that are cool; celebrate each other; assume good faith; it's OK to not be OK; fuck {[}elon / trump / DHH / tech-or-politics-adjacent weirdo of the week{]}
\item
  Making things. Remaining independent.
\item
  Knowledge \& awareness; earnest love of the weird and silly; thoughtful and playful uses of technology (esp.~The Internet); mutual respect and care
\end{itemize}

\section{Inclusivity and Openness}\label{inclusivity-and-openness}

Words people used to describe the community around the idea that it's open and inclusive. They also used words like: respect, friendly, helpful, encouragement, inviting, welcoming, inclusiveness.

While the words inclusivity and openness were used, some also described the community being closed as part of this inclusivity. Some called it safe, a safe haven or safe space for expressing thoughts. Additionally others talked about it being also a place where there was privacy and also confidential-ish.

Quotes around inclusivity and openness:

\begin{itemize}
\tightlist
\item
  Openness, curiosity, fun, belief that the internet can be a wonderful, creative place.
\item
  Safe spaces, where people feel comfortable to be their true authentic selves. Inclusivity, where everyone is welcomed, seen and heard. Encouragement and support in all aspects of life, from local classifieds-type of sharing, to deeper and more personal concerns.
\item
  having a safe community that supports each other
\end{itemize}

\section{Curiosity/Intelligence}\label{curiosityintelligence}

Words people used to describe the value of curiosity include: insightful, interest, knowledge sharing, news sharing, resource sharing, resourcefulness, smart, thoughtful discussion, Curiosity, Interesting, learning, understanding, exploration. As we'll see later, many folks said that the community was smart and that they appreciated the level of intelligence in discourse.

Quotes around curiosity and intelligence:

\begin{itemize}
\tightlist
\item
  Super bright people who are really nice and very tech savvy
\item
  Smart creative folks sharing and helping each other.
\item
  the confluence of curiosity, expertise and community
\item
  low-stakes learning community
\item
  Thoughtfully sharing highlights of the internet, commentary, empathy, joy, and advice
\item
  Interesting news, events, and discussions that I don't find anywhere else
\end{itemize}

\section{Kindness and Empathy}\label{kindness-and-empathy}

The value of kindness and empathy took many forms: compassionate, helping, a caring community, connecting and nurturing, gentle, kind, wholesome-ness, earnest, empathy, integrity.

Quotes that discuss kindness:

\begin{itemize}
\tightlist
\item
  Supportive \& honest, and a kind place to be on the internet.
\item
  Empathy. A safe space to act like a social network without worrying about the bad parts of the internet.
\item
  Compassionate, empathetic, engaged community caring for each other's creative pursuits to become more whole humans.
\item
  Creativity and empathy driven kindness
\item
  Empathy and kindness centered willingness to be hyped up about almost anything
\item
  caring and helpful
\item
  Empathy and creativity. We are people who have been brought together by caring about making things thoughtfully in community with others, and that if you are emotionally vulnerable about the difficulties of that journey it will be paid back with kindness and a sense of unity
\item
  Kindness, patience, and passion. Also, showing or trying for best practices for lots of things --- inclusion, managing and sharing emotions appropriately (the \#bad-attitude channel), the reactji people added
\end{itemize}

\section{Emotional Intelligence and Integrity}\label{emotional-intelligence-and-integrity}

Folks seemed to convey a value in the community around emotional intelligence, though it wasn't always named this. Some example phrases: acknowledgment of feelings, earnest good vibes, listening (before fixing). Getting consent, space for feeling feelings and witnessing others, meta communication, communication as a skill, keep the bad vibes in their own channels,

Similar sentiments could be found in a sense of integrity in the community such as words like Thoughtful, Respect, Respectful, civility, honesty, humanity, principled, respectful, sincerity.

Quotes:

\begin{itemize}
\tightlist
\item
  Everyone is valuable and deserves to be treated with respect.\\
\item
  Acceptance of others and their ideas. Breadth and diversity of userbase.
\item
  Mutual support/big sense of community , interest in others ideas, joy of sharing and supporting, doing our best to make sure everyone is thoughtfully heard and included
\end{itemize}

\section{Overall Values Statements}\label{overall-values-statements}

Some folks shared phrases that seemed to be more general and to encompass several of the values above.

\begin{itemize}
\tightlist
\item
  Collective hive-mind of support, creativity, appreciation for art with a set of overlapping societal principles that steer healthy moderation and a general openness to new dialogue or change.
\item
  Be the internet you want to see in the world
\item
  a commitment to the good parts of the web
\item
  A like-minded community conscientious of etiquette, while encouraging curiosity and discovery without competing for popularity or ostracizing members.
\end{itemize}

\section{Additional Quotes/Comments}\label{additional-quotescomments}

\begin{itemize}
\tightlist
\item
  I don't know that there are strong core values that come from the slack itself. I think it is in many ways a reflection of the event rather than it's own community with a sense of identity
\item
  To me, it was more than just a companion piece to the physical festival. I attended a few times, regretted not going the first year, and missed out on securing a spot in the last year. It was the crossroads of art, culture, and techology from a group of what felt like inclusionary, progressive and small ``l'' liberal people that agreed to operate under a code of conduct that emphasized politeness and kindness in a time when those things are not the norm (online or in public).
\item
  Progressivism, humanism, and a love for all things indie
\end{itemize}

\chapter{Appreciation For the Community}\label{appreciation-for-the-community}

The question we'll discuss in this section was ``What do you appreciate about the XOXO Slack community?'' Unsurprisingly, this mirrored much of what was discussed in the values section, but with deeper explanation.

\section{The People}\label{the-people}

By far, the overwhelming response to this question of what folks appreciate about the Slack is the other people in the Slack.

\begin{itemize}
\tightlist
\item
  I feel like I found my people here.
\item
  The caliber of people and the fact that it's about as safe of a space as I could imagine existing online. It might actually be the safest space online.
\item
  I appreciate the opportunity to connect with all the wonderful people in the community. This group of intelligent, connected and, attentive people have enriched my existence in many ways, including just talking about important or interesting things in a relatively safe environment.
\item
  so many incredible people, so willing to help each other
\item
  The people, the various emoji that say a thousand words, the different channels for different topics, the thread-first approach to communicating, the organisation of everything thanks to folks like the moderators, and even the very active members I recognise!
\item
  everyone is so awesome and kind
\item
  the people, the thoughtful responses
\item
  Local subject matter experts, friendly people, good discussion
\item
  People who are like me, openly sharing who they are, sharing advice, support, empathy and making every experience shared and a little less hard
\item
  The lovely humans that make up the community and how they interact with each other
\item
  It's an amazing place of creative people who are generous, kind, and a lot of fun. Also Andy Baio finds the best links.
\item
  The ability to assume truthfulness or good faith in the things that are shared, and the ability to be possibly wrong, possibly clumsy, possibly human
\item
  The way everyone embodies the value above and leads through their actions
\item
  the people, the content, the kindness, the attitudes, and the opportunity for closer conversations with lovely people
\item
  The XOXO Slack community is helping and giving. People care. People are enthusiasic about things! People are really smart and interesting!
\item
  People's willingness to be vulnerable and to hold space for others to be vulnerable
\item
  All of those core values! And the sheer interestingness of the members of the community
\end{itemize}

\subsection{Diversity of Perspectives}\label{diversity-of-perspectives}

One of the attributes of the people that folks commented on was how many diverse perspectives they felt could be found in the slack.

\begin{itemize}
\tightlist
\item
  The people who are on it, come from a lot of different perspectives and backgrounds and that makes it more interesting
\item
  The variety of backgrounds and perspectives of participants
\item
  Diversity of perspectives, freely given knowledge, empathy and compassion for experiences, range of interests, learning and growing together
\item
  Such a breadth of experience and diverse skills in the members!
\item
  That it's far more of a diverse community --~along almost any axis, other than perhaps political leanings --~than I would ever have access to in my day to day life.
\item
  diversity, empathy, endless curiosity
\end{itemize}

\section{Resources}\label{resources}

\subsection{Asking for Information}\label{asking-for-information}

Respondents discussed using the Slack to share knowledge, ask for help and to learn from one another. They described people being generous and helpful when they did so.

\begin{itemize}
\tightlist
\item
  I love having a place to keep up to date with links and stories shared in channels plus a place to connect and ask questions on a variety of topics.
\item
  When I have a question about damn near anything (resource, fact, local stuff, whatever) someone knows the answer and is likely to be similar enough in mindset to me that I can trust it. And, the goofy shit I like to share always seems to find an appreciative crowd
\item
  Breadth of channels. I can ask/answer Maker or home-improvement questions, keep up to date on tech/politics, and see folks being happy about stuff they made or found.
\item
  People's openness to help, share knowledge, and generally be excepting
\item
  When I have a question about damn near anything (resource, fact, local stuff, whatever) someone knows the answer and is likely to be similar enough in mindset to me that I can trust it. And, the goofy shit I like to share always seems to find an appreciative crowd
\item
  Being able to ask all the dumb questions about anything and everything, and having really smart, capable people reply with their wisdom and resources
\item
  Ability to ask kind individuals both hard and easy questions across a variety of topics
\item
  People who are like me, openly sharing who they are, sharing advice, support, empathy and making every experience shared and a little less hard
\item
  Diversity of perspectives, freely given knowledge, empathy and compassion for experiences, range of interests, learning and growing together
\item
  Feedback and support
\item
  Quality information
\item
  I appreciate the broad range of skills and expertise. I also love that we all are passionate about something and we're open to hearing about other's passions.
\item
  the ability to drop in and engage, the willingness of the community to provide support (responses, engagement, advice, etc.) and the depth and breadth of knowledge across all its members
\item
  I like that I have a place to hear about the things I'd hear about on social media (news, tech news, neat creative projects, dumb jokes) without the context collapse of social media, and in a moderated and kind environment. As a curated microcosm of the internet at large, I can dip my toe in and feel rewarded without getting hit with the anxiety and overwhelm - and the smaller scale means I've made some lasting friendships and connections because I'm able to show up more fully than I would if I was more afraid.
\item
  private identity channels; knowing that i can ask basic questions and people love info-dumping with no judgement
\item
  It's a large group of smart, successful, creative, people who as a whole, is good-internet. I've made friends and professional connections here and my favorite by-product is the community's ability to help me gather information on almost anything, which also spoils me because I'm not confident in my abilities to find reliable information through traditional web searches.
\item
  SO much. It's a place where I learn about and discuss current events and art with folks I like and admire. Also, frankly, it'a a place I can lurk! I expreience severe rejection sensitivity in online spaces, so the fact that I can pop a little emoji next to a post to show sympathy, curiosity, etc is really nice for me. I also experience the moderation as very present, compassionate, and firm, which REALLY makes a difference in the space.
\end{itemize}

\subsection{Learning}\label{learning}

Folks also discussed how important it was to learn from others in the slack and how much they had learned over time. People talked about how smart other folks in the Slack are and how intelligent the conversations were in the channels. They also described the space as being curated and having a variety of things they wouldn't otherwise see/hear about.

\begin{itemize}
\tightlist
\item
  I learn SO much. There's so many smart voices.
\item
  The smart creative, mostly likeminded folks.
\item
  Local subject matter experts, friendly people, good discussion
\item
  Honestly, the ``closed'' nature of the community worked for me for professional reasons. But I also loved the diversity of topics and how it let me get to know different people from the general community.
\item
  Being able to ask all the dumb questions about anything and everything, and having really smart, capable people reply with their wisdom and resources
\item
  From the delightful shared experience of the Animal Crossing channel in early Covid, to the freewheeling discussion of politics, to the creative problem solving and solidarity of many other channels, questions, learning and growth are supported and so are the personal connections we've forged at the festival and in this digital space.
\item
  empathy, wisdom of the crowd, discovery
\item
  Diversity of perspectives, freely given knowledge, empathy and compassion for experiences, range of interests, learning and growing together
\item
  knowledge base
\item
  Curious, smart, politically progressive
\item
  In depth and well meaning discussions on difficult topics - seeing people's various opinions and considerations help me form my own
\item
  A large amount of very smart people all feeling like it's a safe place to express their thoughts
\item
  It's a large group of smart, successful, creative, people who as a whole, is good-internet. I've made friends and professional connections here and my favorite by-product is the community's ability to help me gather information on almost anything, which also spoils me because I'm not confident in my abilities to find reliable information through traditional web searches.
\end{itemize}

\subsection{News}\label{news}

Folks also described the slack as the place that they used to find the news, and that they relied on each other to have compelling and interesting conversations about topics that were on the front page of the newspapers.

\begin{itemize}
\tightlist
\item
  a way to get news, to get fresh and useful perspectives on news (politics and tech, mostly)
\item
  Great source for news \& interesting projects from all around the internet, fantastic discussions with smart people who are experts on various topics
\item
  Resources in news/likeminded spirit
\item
  Discussions on the XOXO Slack are always my first choice for any online interaction, it's replaced all news sites
\item
  Excellent moderation, ability to check in for local Portland events plus national events (like politics and C19). My first go-to community for EVERYTHING, including classifieds.
\item
  Very up to date news, an active community, lots of channels dedicated to very specific topics
\item
  It's a place where I can stay abreast on the difficult topics of the modern digital world and remain connected to other digital artists without the terrible emotional pain that comes along with using social media like Twitter
\item
  Surfaces good posts and news (replaced social media). People there make me feel seen. We don't ramble, we share and support.
\item
  I don't like to consume news because of its ill effect on my mental health, and this Slack helps me stay engaged with issues I care about. I appreciate hearing the perspectives of this community.
\end{itemize}

\subsection{Cool Stuff People Share}\label{cool-stuff-people-share}

People described the benefit of being in a space where folks share cool stuff and they get to have curated content that they like.

\begin{itemize}
\tightlist
\item
  Being able to stay in touch with friends made at XOXO. Seeing and hearing about all the cool stuff people make and share, whether from within the community or elsewhere. Recommendations on local and online businesses for specific needs.
\item
  Great source for news \& interesting projects from all around the internet, fantastic discussions with smart people who are experts on various topics
\item
  Curated content, exclusive access, there are experts in everything
\item
  Fellow nerds and weirdos who read the internet so I don't have to. Also I like that sharing is encouraged. I'm not someone with any real online presence so it's nice having that space. Also, being able to have people productively disagree, though there have been a few instance of that being tested in the politics channel, the way it was handled by the Andy's was I think very good.
\item
  empathy, wisdom of the crowd, discovery
\end{itemize}

\section{Shared Values}\label{shared-values}

Many folks talked about how they shared values with others in the community, including the upholding of the code of conduct, the sense of political alignment and a common belief in the idea that folks should be acting in ways that are not harmful.

\begin{itemize}
\tightlist
\item
  it's a community with a shared \emph{ethos} rather than shared subject matter, large enough to have lots of casual relationships/interactions but not the entire goddamn internet
\item
  It's the only social network that feels truly connected and wholesome. I use it to connect with a community of people that share my values, and know that the people sharing information care about the quality of their information, and will always act in the best interest of the community and each other. And if anybody acts in harmful ways, they will learn from their mistakes and act differently (or be removed from the community)
\item
  General political/social alignment, joyfulness, creativity, persistence of identity
\item
  A group with diverse backgrounds and experiences but shared values
\item
  Varied perspectives, but somehow lots of people with the same morals as you. Easy to make friends.
\item
  People who are like me, openly sharing who they are, sharing advice, support, empathy and making every experience shared and a little less hard
\item
  it's a collection of nice people who do interesting things, the whole community is invested in upholding the existing CoC/culture, the word ``building'' comes to mind (community/each other up/cool stuff/process/etc.)
\item
  Nearly everyone visible on Slack* appears committed to the core values of the Slack, providing a safe space for others, enforcing norms to that end, and politely accepting correction from others. There is a general understanding that not all behavior or speech can be welcome and still offer the safety needed to help people be and grow true to themselves. (*Behavior outside of this range may be happening and is not visible due to vigilant moderator efforts.)
\item
  It's the only social network that feels truly connected and wholesome. I use it to connect with a community of people that share my values, and know that the people sharing information care about the quality of their information, and will always act in the best interest of the community and each other. And if anybody acts in harmful ways, they will learn from their mistakes and act differently (or be removed from the community)
\item
  Learning from the breadth of knowledge and experience of my Slack pals in a place where we tacitly acknowledging shared beliefs that I can't assume are shared everywhere (caring for other people, respecting other people's experiences and privacy)
\end{itemize}

\subsection{Kindness}\label{kindness}

Many, many people in the survey described the community as kind. Some examples:

\begin{itemize}
\tightlist
\item
  I have a group of kind, caring humans who give a shit.
\item
  Open minded people, kindness and empathy centered, a bastion of covid-awareness.
\item
  I appreciate how people are supportive and share their thoughts
\item
  Positivity, curiosity.
\item
  People who are like me, openly sharing who they are, sharing advice, support, empathy and making every experience shared and a little less hard
\item
  The kindness and enthusiasm
\item
  Same thing! All the creative, kind people. The commitment to inclusivity and authenticity
\item
  everyone is so awesome and kind
\item
  The kindness, the knowledge, the generosity of spirit of everyone involved.
\item
  everyone is so nice and we assume best intentions! also each account is one real person who has presumably interacted with the other people here IRL at least once, as part of a larger network (which lends some accountability/responsibility)
\item
  People are nice to each other. There are regular characters. The usual Slack conveniences that make it possible for people to not repeat themselves and to let me ignore US politics if I want to.
\item
  the small acts of kindness or indications of support people offer each other; emoji indications or threaded replies have genuinely lifted my spirits
\end{itemize}

\subsection{Safety}\label{safety}

One of those shared values seemed to be safety. People consistently referred to the slack as a safe space, or a place were they felt they could be themselves. They also commented that this made the space trustworthy and easy to be in.

\begin{itemize}
\tightlist
\item
  people assume good intent and read with compassion and reply with civility; it's a safe space for people with marginalized identities, people are funny and kind and creative and there's very little `rock star' attitude (even from people who are literally rock stars)
\item
  The safety and established trust.
\item
  The caliber of people and the fact that it's about as safe of a space as I could imagine existing online. It might actually be the safest space online.
\item
  It seems like both a safe place for optimism and constructive criticism.
\item
  A safe place for me to land for professional and personal needs
\item
  I love having a place full of people I respect and admire, where I nonetheless feel safe being honest and vulnerable.
\item
  It is consistently the kindest space I'm a part of. It's also the largest space I'm a part of that feels private and provides safety in its privacy.
\item
  Such a thoughtful space! I do not know how much moderation it requires, but from the outside it is a smoothly functioning space full of independent-minded folks, which is no small thing! I feel safe and cared about, even by relative strangers.
\end{itemize}

People also described this as a place with people they could trust

\begin{itemize}
\tightlist
\item
  My views align with those around me, and I enjoy chatting with people who share common ground. Since we all respect the code of conduct, there's less negative behavior, making interactions smoother. I value learning from the random, niche interests everyone brings, as someone always has expertise in something I'm curious about. These diverse perspectives help me expand my understanding and knowledge.
\item
  The broad spectrum of experiences and knowledge so that I can find out about a lot of different topics, from people who I feel like I can trust because they've been vetted.
\end{itemize}

\subsection{Privacy}\label{privacy}

Some folks linked safety with the privacy of the space.

\begin{itemize}
\tightlist
\item
  It is consistently the kindest space I'm a part of. It's also the largest space I'm a part of that feels private and provides safety in its privacy.
\item
  Honestly, the ``closed'' nature of the community worked for me for professional reasons. But I also loved the diversity of topics and how it let me get to know different people from the general community.
\item
  The wide world will never see it so I can really be myself
\end{itemize}

\subsection{Community}\label{community-1}

People discussed community as something that they felt a part of, that was important to them and that they shared with others.

\begin{itemize}
\tightlist
\item
  It's a very welcoming community, there's a shared sense of how hard it is to make things and an appreciation of what people do
\item
  It gives me a hometown, which I haven't had in ages and ages
\item
  I like how people don't jump to knives at the first opportunity, and appreciated that an Andy would step in and lay down the kindness law, gently then firmly, if needed, while not putting down one side or the other. I also like how people are able to browse and find others who have similar interests by looking at the channel list. I like that there are private spaces, as well. I also appreciate the `benevolent dictators' both the structure but also that our dictators are kind, patient, and passionate.
\item
  The ease of re-engaging after varying levels of activity, but also a core group of active members. This is the only only community I've belonged to where I feel most comfortable expressing opinions and not competing for popularity. In other social networks, I'm often forced to create separate identities for friends/family, private, and professional audiences.
\item
  the ability to drop in and engage, the willingness of the community to provide support (responses, engagement, advice, etc.) and the depth and breadth of knowledge across all its members
\item
  it's a collection of nice people who do interesting things, the whole community is invested in upholding the existing CoC/culture, the word ``building'' comes to mind (community/each other up/cool stuff/process/etc.)
\item
  it sits perfectly at the intersection of art, technology, nerdery, and social good. there are plenty of communities that bridge two of those things but I haven't found another that covers off on all of them like XOXO does.
\item
  The ability to communicate with others across multiple points of interest and the opportunities for mutual aid through ideas, concepts, and real world help.
\item
  Connection. This community has gotten me through some really hard times and while I may not know many of the folks here in person, I consider many of them friends.
\item
  \#portland and \#classifieds for local connection, \#good-internet / \#politics-and-news / \#tech-culture to keep up with relevant news and events
\item
  Ability to vent and then find distractions and cool things and inspiration
\item
  So many things. It's a place to go with wild ideas, or burning questions, or amusing anecdotes. It is a place to be supported and to give support.
\end{itemize}

\subsection{Openness}\label{openness}

People also described the community as open:

\begin{itemize}
\tightlist
\item
  People's openness to help, share knowledge, and generally be excepting
\item
  Open minded people, kindness and empathy centered, a bastion of covid-awareness.
\item
  Openness, creativity, sharing and support
\item
  that it's open, helpful, welcoming, entertaining
\end{itemize}

\section{Portland}\label{portland}

Folks in Portland in particular talked about how useful the slack was to find things in their community.

\begin{itemize}
\tightlist
\item
  I'm a lurker, more than a poster. The \#portland slack has been incredibly valuable as a local. I've attended events, visited restaurants, learned about civic activities, and donated to causes I learned about via Slack.
\item
  For Portland especially, xoxo is a wonderful resource of news, vendors, events.
\item
  Its relative privacy and safety, most of the cool new things I've learned about in Portland and on the Internet were via this Slack.
\item
  It's been a really good resource for things going on in Portland Oregon! Otherwise it's just nice to see what people are up to or reading.
\item
  feels like the small room of good internet circa 2010-2015 and/or like Twitter before it got bad. Secondly, as a new native to Portland I really appreciate participating in a community of like-minded local creators
\end{itemize}

\section{Alternative to Media}\label{alternative-to-media}

Some folks talked about the community as a place where they could have a discussion that wasn't as broken as some of the media we have now.

\begin{itemize}
\tightlist
\item
  It's a group that cares about and wants the best from the Internet, but without the blind techno-Utopianism of mainstream tech culture.
\item
  I like that I have a place to hear about the things I'd hear about on social media (news, tech news, neat creative projects, dumb jokes) without the context collapse of social media, and in a moderated and kind environment. As a curated microcosm of the internet at large, I can dip my toe in and feel rewarded without getting hit with the anxiety and overwhelm - and the smaller scale means I've made some lasting friendships and connections because I'm able to show up more fully than I would if I was more afraid.
\item
  It's a place where I can stay abreast on the difficult topics of the modern digital world and remain connected to other digital artists without the terrible emotional pain that comes along with using social media like Twitter
\item
  A digital shelter from a capitalistic web
\item
  It helps keep me centered in a greedy, competitive world.
\item
  nice place for nerds to nerd out
\end{itemize}

\chapter{Concerns}\label{concerns}

We asked folks ``What concerns (if any) do you have as the XOXO Slack winds down?''

\section{Losing the Community}\label{losing-the-community}

People were very concerned about losing the community that was built.
* Significant loss of community and support
* I am concerned that I will lose track of folks and the important, silly and entertaining chats we have
* Losing touch with the community
* We will lose people.
* It will be hard to find ``my people'' again in other places
* The lack of community that has a baseline of trust good enough to bring its users forwards. I won't have people to bounce ideas off of who I know share my values, I won't be able to look for jobs with people that share my moral code. It's tough.
* The community won't continue, I'll no longer be in a vibrant internet community
* Losing connection and community --- I'm kind of terrified of what my world will look like without this space (but I definitely understand and respect the decision).
* massive dropoff in participation/access, potentially less user-friendly new platform, increase in anonymity, potential decrease in community trust/network
* I stumbled upon the after-slack channel kinda randomly and had no idea it existed or that the slack would be winding down, and I'm kinda surprised there hasn't been a post in \#announcements about it to get more people aware and involved.
* That I won't be able to engage with a community I've come to appreciate and speak/discuss things in a friendly and safe space.

\section{Fragmentation}\label{fragmentation}

Folks mentioned worrying about the splintering, fracturing or fragmenting of the community.

\begin{itemize}
\tightlist
\item
  Fragmentation of the community. Or a replacement that doesn't allow for the same breadth of topics and discussions. Discord sucks at threading, and threaded topics keep a lot of the xoxo slack legible and approachable. I just don't want to lose these people, or the quality of the discourse we have.
\item
  I worry that this community will fracture in a way that will eventually fizzle out. We all care where this goes, but that doesn't necessarily make it easy.
\item
  The community splintering a bit (inevitable) as some point just won't migrate to certain platforms probably
\item
  Splintering of the groups into isolated subgroups that are harder to find.
\item
  I really think that we'll never be as central as we are now; we're going to scatter and lose track of each other
\item
  The community fragmenting into many different new spaces and it being difficult to know where everyone went
\item
  I think the community will scatter, so it's not really a concern, but it's something I'm sad about.
\item
  I'm worried the community will fragment and it won't be as diverse and full of a wide variety of experiences and viewpoints.
\item
  Fragmenting the community, not replicating the x-channel experience elsewhere, fear of choosing an under-supported chat/forum solution and running into maintainence difficulties and seeing the community die out
\item
  Fears: 1. the community scatters to the wind without a clear locus 2. community gets fragmented 3. no other such community exists /cannot be replaced 4. do not want to lose this both local \& extended community
\end{itemize}

Similar to the worries about fragmentation were folks worried about the lack of coherence in a new setting.

\begin{itemize}
\tightlist
\item
  I feel concerned that a group won't cohere in a new place, and that values and conduct will be harder to manage without the same leadership.
\item
  Maintaining connections with people from this group without having to join a lot of small things in different places
\item
  I don't want to lose the community we created
\end{itemize}

\section{Loss of Contacts/Connections/Networks}\label{loss-of-contactsconnectionsnetworks}

People also worried about the loss of the people they know, and the contacts and network that the community is made of:

\begin{itemize}
\tightlist
\item
  that we'll all scatter to the winds
\item
  losing touch with locals and those I rarely see
\item
  I really rely on hearing from my friends in my little pane of glass every day, I don't know what I will do w/o them
\item
  Losing regular contact with the community who shared those core values
\item
  I really appreciate this tiny corner of the internet and all the voices; will be sad if it goes away completely!
\item
  disintegrating connections/strong as well as distant relationships, lost art and humour and logs of supportiveness, loss of casual but trusted and confident advice and recommendations
\item
  Just that I'll miss every one.
\item
  I'll miss people (en masse), though I don't know anyone well enough to actively keep in touch
\item
  more work to connect with people outside the Slack
\item
  losing touch with people I've met, losing the breadth/diversity of content and opinions. I love how many amazing things I have discovered beit art or technology projects or information about local activities that I wouldn't have known about if not for this community.
\item
  Losing that incredible social network both in terms of resources but also in just enriching my daily life
\item
  Speaking entirely selfishly, I'm worried that I will lose contact with a network of artistically successful, like-minded, and supportive people. I'm not a member of other communities so capable of helping me do better work or promote that work, and otherwise connecting me with opportunities that interest me (employment or otherwise). XOXO Slack is both private \emph{and} populated with people I admire in various fields, as well as friends I don't otherwise have contact with. I have IRL friends and other online communities, but no group as diverse or as interesting as XOXO. I can't imagine building such a network from scratch today, especially with the degradation of online social networking tools.
\item
  Losing this community of amazing people.
\item
  i'll miss being able to connect with this group of people, and seeing the fascinating things they share and make
\item
  We'll loose a lot of people and breadth of experience
\item
  Not having a place to vent, losing contact with people I care about, not having a place to process emotions and seek advice and talk about things like ptsd, anxiety, depression, impostor syndrome and also feeling alone when I don't see others going through the same things
\item
  Losing touch with some of my favorite people on in the internet (only a few of whom I've met IRL)
\item
  Missing out on turning those weak connections with peers in channels into far greater friendships.
\item
  I'm concerned about losing touch with people, and about losing an active stream of content/conversation
\item
  Losing track of everyone and all of the great resources and conversations. Losing the job board.
\end{itemize}

\section{Loss of Resources}\label{loss-of-resources}

Folks talked about the resource of the community and their fears about losing those resources.

\begin{itemize}
\tightlist
\item
  I'm worried I might lose some of those resources and community at a time when online spaces already seem to be shrinking.
\item
  Losing a source of important things happening in the world, losing a source of positive information.
\item
  I fear that I will lose access to an important source of information and I'll end up having to stay up to date in stressful spaces
\item
  losing the incredible resources of this community
\item
  Keeping all that varied input together
\item
  Losing all of the incredible resources/recommendations that I've saved there as well as all the people I've connected with who I'm maybe not connected with on other areas of the internet
\item
  The splintering of the community and loss of archive. The strength of the community comes from the interactions between folks. I explore the slack daily and learn so much and interact as much as my stamina can handle. Sometimes often, sometimes not. but! The variety of opinions and conversations is so important and ensuring there's minimal friction for folks to engage is important. I also search the archives for help and having a persistent record is useful to me.
\item
  disintegrating connections/strong as well as distant relationships, lost art and humour and logs of supportiveness, loss of casual but trusted and confident advice and recommendations
\item
  Loss of creative ballast and diminished trust in humanity.
\item
  Not being able to connect to people whose opinions, feedback, and camaraderie I've grow to trust and rely on
\end{itemize}

\section{Loss of Special XOXO Spirit}\label{loss-of-special-xoxo-spirit}

There was also a sense of loss of the intangible sense of the spirit of this community

\begin{itemize}
\tightlist
\item
  Initially, the new space doesn't \emph{sufficiently} capture the spirit. Long-term, that it degrades.
\item
  While it's probably inevitable that a platform shift will cause some people to drift away/not come with, I am hoping there is something that reflects the same core values of the Slack community in a sustainable (financially, emotional labor-wise) way
\item
  This Slack also had a subtly reinforced etiquette amongst active users, like threading topics, redirecting to niche channels, or contextualizing shared links. I fear a new platform will drive a wedge between active core users, seasonal festival users, and new non-XOXO members. Each channel also differs in etiquette and moderation, which can be lost in translation when migrated.
\item
  Losing people, losing the intangible thing that exists on the XOXO Slack
\item
  Keeping the same community and thoughtfulness together on whatever new platform
\item
  that the magic will dim; this is reliably my favorite place online
\item
  Losing people, losing the intangible thing that exists on the XOXO Slack
\item
  Scared we won't catch this vibe again, and will end up with a rump-xoxo slack that is somehow sad
\item
  Losing instances of kismet or good luck which occur when a very large group of talented people are all drawn to the same place - there's a kind of magic to being surrounded by people who are also devoted to both kindness and trying their best at what they do. There are not many places in the world where I could say as an artist, ``Randomly, I'm obsessed with these repeat-pattern window films'' and someone says ``Oh, my wife is half of that design team!'' and now we're in touch and might get to make cool stuff
\end{itemize}

\subsection{I'll Never Find Something Like This Again}\label{ill-never-find-something-like-this-again}

Among the ways that folks talked about their fear of losing the special spirit was with phrases around how hard it would be to find something like this again.

\begin{itemize}
\tightlist
\item
  I will never find such valuable community
\item
  I'll miss it and it won't be replaced with something else.
\item
  I really don't know anywhere else like this :(
\item
  I don't know where I'm going to find another similar community, or to find these people elsewhere!
\item
  I'm just worried I won't be able to find the next spot!
\end{itemize}

\subsection{This Slack is Different}\label{this-slack-is-different}

Several folks also remarked that this place is different than other types of social media and so it cannot be easily replaced with another platform or group of people.

\begin{itemize}
\tightlist
\item
  I'm going to lose all the friends I haven't made yet! And I don't want to be on public social media.
\item
  The cross discipline nature of the whole group adds so many perspectives and variety to conversations. More topic-focused successors might lose that.
\item
  losing insights on tech that aren't awful and are instead optimistic, losing local (Portland) knowledge base
\item
  Loosing instant access to a wonderful community of people that has filled the social media hole in my life after deleting everything but LinkedIn.
\item
  I am just so sick of everybody-has-a-megaphone performative-outrage social meida sites.
\end{itemize}

\subsection{Slack as a ``Third Space''\,''}\label{slack-as-a-third-space}

Other folks talked about the slack as a sort of ``third place'' or a spot that one returns to. Some folks talked about it as a water cooler or a place to hang out.

\begin{itemize}
\tightlist
\item
  I really rely on hearing from my friends in my little pane of glass every day, I don't know what I will do w/o them
\item
  Lose touch with people, lose my ``third place'' where I have felt the most comfortable online.
\item
  I work from home and it's my main watercooler! Will miss that and hope the replacement functions similarly.
\item
  It's my favorite place on the internet
\end{itemize}

\section{Concerns about New Platforms}\label{concerns-about-new-platforms}

Some folks shared their concerns about new platforms

\begin{itemize}
\tightlist
\item
  Future loss or disassociation from the historical community record, challenges with invitation/openness to new users on a new platform
\item
  i'm not going to use discord and that seems like it will be where yall go
\item
  Keeping the same community and thoughtfulness together on whatever new platform
\item
  It was my favourite Slack for sure, but I don't think it necessarily has anything to do with the technology. It could have been a web board or a Reddit or whatever made the moderation and the delicate balance of inclusion and exclusion work. I would be excited to join another or multiple communities that spawn from here based on their mandates. Being able to talk about culture and technology and art are important to me and the lack of geographical boundaries benefitted me living outside of a major US city.
\item
  Which platform we end up on. Will a free tier slack work for us? Will discord's gamer stink scare people away? Will a third option for our needs? It's hard to know!
\item
  Having to learn to navigate a new non-slack on line community. I like Discord.
\item
  That the community has an easy transition to a new space, without unnecessary technical barriers, and that whatever comes next (at least initially) remains as a closed community.
\item
  I really dislike Discord and am worried we'll either get fractured into so. many spaces we'll lose the magic that exists on Slack or that we'll move to something I'm not comfortable with and don't want to be on. * I worry we will end up on some corporate platform and be in the exact same situation in a few years
\item
  I want us to find a new home that can stay as dynamic and vibrant. That can be appreciated in both real time and periodically.
\end{itemize}

\section{New Leadership}\label{new-leadership}

Some voiced concerns about whomever assumes the leadership of what comes next.

\begin{itemize}
\tightlist
\item
  This grew organically and thoughtfully as an outgrowth of the conference. I worry about platform, lack of Andy's-ness.
\item
  I'm a little worried about new leaders making decisions that drive folks away, but I'm much more worried about nobody taking the reins and everyone slowly fragmenting and dispersing.
\item
  Super tough for whoever decides to take on volunteer responsibility
\item
  We've been able to operate a digital space under implicit values and norms established by the IRL event. Without the Andys as our benevolent dictators (even if in actuality much/most Slack moderation wasn't done by them), and potentially without the context of IRL XOXO if we eventually let in outsiders, recreating the same psychological safety will require more work. I'm also concerned about community attrition; an ``unofficial'' thing without an annual event may be more difficult to maintain healthy numbers (which ties into potentially bringing in new members)
\end{itemize}

Some discussed concerns around governing with a new leadership

\begin{itemize}
\tightlist
\item
  how do we govern the new space? how to avoid a single mercurial owner-member shutting it down abruptly? can we pay folks to moderate/manage the community? or is it something like a nonprofit board? this community is worth paying for.
\item
  Governance, if building trust in the new decision makers
\item
  What comes next? How should it be similar and different? Will I be allowed into the next thing? If something new spins up, how could I contribute to it's sustainability and success?
\item
  Retention and activity with inevitably drop, members may place unrealistic or high demands to match the current level of moderation on the new team
\item
  I think charismatic leadership/tone setting and moderation should be separate functions, as they require two different strengths; one requires openness and vision, while the second requires more rigidity and control.
\item
  how do we make decisions as a group?
\item
  The xoxo community belongs to the community. How can these decisions be make without the community
\end{itemize}

\section{Needs for Types of Community Members}\label{needs-for-types-of-community-members}

\subsection{Mental Health and Well Being}\label{mental-health-and-well-being}

Some folks in the community talked about their mental health being greatly enhanced by the community and their fears about the loss of this community going forward. They said that the Slack combatted their loneliness, their doomscrolling and that it contributed to their well being. They were concerned that their lives would be worse without it.

\begin{itemize}
\tightlist
\item
  To lose the great connections, inspiration, advice and live a lonelier life
\item
  loss of community, disconnection, isolation
\item
  Losing a community that is a vital pillar of my mental heath with no idea how to replace it if we don't figure out a way to keep it going.
\item
  Not having a place to vent, losing contact with people I care about, not having a place to process emotions and seek advice and talk about things like ptsd, anxiety, depression, impostor syndrome and also feeling alone when I don't see others going through the same things
\item
  I doomscroll again
\item
  I'm very concerned about losing this space as the internet becomes more hostile.
\item
  I feel very socially isolated right now and this going away is going to make that worse
\item
  The XOXO Slack has been a valuable resource for me, both as a news source (as someone who doesn't spend any time on public social media) and as a community. It has been especially valuable during the pandemic. I would really miss having that connection to a group of similarly pandemic-cautious people to remind me that it's the rest of the world that's making wild choices, not us.
\end{itemize}

\subsection{People who Lurk}\label{people-who-lurk}

Folks specifically brought up the topic of folks who ``lurk'' in the slack and don't post much.

\begin{itemize}
\tightlist
\item
  I imagine there are a lot of lurkers who still get a lot of value. Personally I actually add to a conversation like once a month maybe? And I'm worried that any sort of ``oops it all fragmented into a million communities'' will maybe impact those folks the most. Though maybe part of community is participating, so idk.
\item
  the xoxo community (and slack) is made up of all of the people, lurkers as well as active posters. I worry that the new community will only have currently active people who will try to attain a kind of nostalgia instead of allowing it to move forward in a way that preserves our shared ideas of community
\end{itemize}

\subsection{Newcomers}\label{newcomers}

Some respondents highlighted the difference in responses for those who are newer to the community:

\begin{itemize}
\tightlist
\item
  being new in 2024 means less of a `loss' for me personally, in terms of a history with the slack, but that belies a concern that longer or more tightly-knit cohorts that have existed throughout will splinter off and those newer folks may not have that or as easy an ability to catalyze a transition as those older folks with more historical relationships in the community, so an understanding of that and some way of making sure there isn't so much slipping through the cracks if there doesn't have to be
\item
  The thing is that I want this to be an opportunity for growth, including opening up invitations eventually. I know not everyone will make the jump, and even less will be frequent users, but I really hope this is also an opportunity to bring in new voices. I was a volunteer this year, and some thing that Andy said was to be mindful of the fact that for many people this was their first XOXO, so don't make it all about sipping on the sweet nectar of your memories. I want the next chapter to be like that, but more. sorry for the long answer
\end{itemize}

\section{Positive Words}\label{positive-words}

Some folks shared words of affirmation or enthusiasm for what comes next instead of concerns

\begin{itemize}
\tightlist
\item
  I am confident that the community wants to continue.
\item
  This is a curated group and allowing new people in will change that in fundamental ways. I think it's okay if the community ``winds down'' (fewer users) because there's still a core happening, and people can come back in as their own needs change.
\item
  I'd be sad to see the community drift apart; so glad there's an effort to create a new one!
\item
  In truth, it will evolve, but never be the same. It's like an unconference, the experience just happens there and just for the people there. It will live on, just in a different way
\item
  I suspect that I will not be as comfortable, sharing more personal things in any other forum. And that's OK. This is giving me an opportunity to reach out to the people they would like to stay in touch with.
\item
  While it's probably inevitable that a platform shift will cause some people to drift away/not come with, I am hoping there is something that reflects the same core values of the Slack community in a sustainable (financially, emotional labor-wise) way
\end{itemize}

\section{Additional Comments}\label{additional-comments}

\subsection{Sadness}\label{sadness}

Some discussed their sadness around the shift

\begin{itemize}
\tightlist
\item
  No concerns really, it's just sad
\end{itemize}

\subsection{Concerns About Safety}\label{concerns-about-safety}

\begin{itemize}
\tightlist
\item
  Privacy and safety. What prevents people from screenshotting and sharing once channels go inactive? What rules will there be for this new era of interaction with the archives?
\end{itemize}

\subsection{I'll Go Wherever}\label{ill-go-wherever}

Some said they would go to whatever comes next:

\begin{itemize}
\tightlist
\item
  Would like to follow the diaspora to other venues
\item
  I don't really care where it goes, as long as it survives.
\end{itemize}

\subsection{Concerns about Channels}\label{concerns-about-channels}

\begin{itemize}
\tightlist
\item
  Keeping various sub-communities (channels) and safety of the space
\end{itemize}

\subsection{Missing the Emojis}\label{missing-the-emojis}

\begin{itemize}
\tightlist
\item
  I'll miss the custom emojis; they were a second language to this community.
\item
  i will feel like i have lost a ton of my working vocabulary if we lose all our reactji! and I will very much miss people who for whatever reason can't transition to whatever comes next
\end{itemize}

\chapter{Impact}\label{impact}

We asked the community: What would you like to share publicly about how the XOXO Slack has impacted your life?

\section{Quality of the Community}\label{quality-of-the-community}

Responses talked a lot about the quality of the community and how the community felt like an example to them of what good community and good internet could look like.

\begin{itemize}
\tightlist
\item
  XOXO Slack is one of the only online spaces where I feel like I'm connecting with an actual community of people willing to give me consideration as a person, where I can try new things, make mistakes, and trust that I'll receive both well-meaning feedback and forgiveness. We can chat and share casually about a wide range of topics without so much concern about a hypothetical future entity using what we say against us disingenuously. I can strive to be a good person, and not just self-censor to where I look like one.''
\item
  This community has changed my life. I've made lasting friendships and received career altering advice and assistance from the generosity and intelligence of the people assembled.
\item
  XOXO slack is hugely important to me as a source of community. Also (and far less importantly) I get a lot of cool links from the slack that I use to make it seem like I'm cooler in other communities
\item
  Xoxo slack provided community during the darkest time of the pandemic (and beyond) . I often feel isolated from creative communities which can be cliquish. Xoxo slack is the best type of inclusive community: bringing people together to learn and grow as responsible creative beings on this planet.
\item
  This community has been the closest to feeling like ``my people'' of anything I have ever experienced. People here have been consistently welcoming and friendly. I learn about things I otherwise would have missed on an almost-daily basis. I feel honored to have this kind of access to so many amazing folks. XOXO and this community have encouraged me to be less shy about sharing things I do and make.
\item
  I have always struggled to find a home on the internet. It's hard to join an existing community, and it's also hard to create one yourself. The XOXO slack always felt like such a welcoming place to just be. I really appreciate what it has been, and will miss it dearly -- but I look forward to whatever is next.
\item
  The XOXO slack is the first place on the internet where I felt at home.
\item
  I have always struggled to find a home on the internet. It's hard to join an existing community, and it's also hard to create one yourself. The XOXO slack always felt like such a welcoming place to just be. I really appreciate what it has been, and will miss it dearly -- but I look forward to whatever is next.
\end{itemize}

\subsection{An Example of Community}\label{an-example-of-community}

Folks referred to the community as something they point to as a good example of what community can be.

\begin{itemize}
\tightlist
\item
  Hard to define because it's been an accumulation of small things that have contributed to my whole identity. I'm not just my job, or my work, or my movements --- I'm also part of this community, this culture. The examples you all set have informed how I try to act and be, both online and offline. And of course the Andys' example of collaboration and leadership will always be a part of everything I do with others.
  It's taught me how to be more inclusive
\item
  this Slack has shaped my thinking about communities and how they work, as well as being an incredible support and resource during a very intense period of my life and in the world. it's one of the only places where I still seek out news and politics, partially because of the ethos of the space. oh! and it's where I found one of my closest co-workers for a couple of years, who is still a very good friend. during a period when so much of the internet became completely unbearable, the xoxo slack remained a constant reminder that good people are doing cool stuff, and that we can all take care of each other.
\item
  Renewed my faith in online community
\item
  It has given me a higher standard of what forums/communities can be. It's helped me keep on writing in my blog even if I don't get a lot of traffic because at least I'm creating something.
\item
  I've said it before but the XOXO Slack has been one of the best internet communities I've experienced, and I've learned so much from the community there. I wish everyone were able to find a community like this one to see what's possible, particularly given the way that social media has become so fractured and hostile.
\item
  It has been my connection to my city and the people within it for so long, I barely know how to deal with it changing or going away. Every day, it's one of the first things I see.
\item
  The XOXO Slack has been a lifeline for me since 2018. It is a safe and caring place I can go for inspiration and information and assistance and a laugh and new music and job postings. I enjoy supporting others in the community. Being in this community gives me confidence in what I'm doing professionally. I don't feel alone. I know that people care and that there are other people like me. I am so inspired and moved by this community and the people I've met here.
\end{itemize}

\subsection{Community Feels Like Old Internet}\label{community-feels-like-old-internet}

Folks often brought up the idea that this felt like the ``old Internet'' or ``Old Twitter''

\begin{itemize}
\tightlist
\item
  It feels like one of the last bastions of ``good internet'' that has roots going back to places like The Well, Suck, etc., and that still lives on, but is increasingly hard to find unless you know where to look.
\item
  It's the only social network I use that feels truly social for social's sake. I trust the community, and love being able to talk with people who share my values, love of the internet, and curiosity about the world and how to make it better.
\item
  as someone who's been terminally online since I was in my teens, I've felt disconnected from my older communities as I've aged. This community feels like my last best connection. It's been the first place I turn in so many moments over the last 5 years
\item
  XOXO Slack is one of my last remaining connections with the spirit of the Old Web and the early weblog community. Optimism, creativity, connection, exploration of the creative and social spaces formed by the tools we use to create and communicate, and sensitive and sober consideration of the social responsibilities of those technologies.
\item
  Since then I've been mourning the loss of my community. Many left with me, but we all scattered to different places. The people in this slack are different (with some overlap) but it really is a community and it feels a lot like Twitter back in the fun days. It has helped me process the loss of Twitter and come to terms with the fact that all communities evolve, change, and dissolve. A community is not defined by the people who are in it, but by the relationships and care that flow between the people. (As in most graphs, edges convey more information than nodes.)
\item
  It was a great way to keep up-to-date with the fest itself but also the communities and topics that interested me. Like ``old'' twitter, I could find a place to co-watch the Apple keynote, for example.
\item
  Social media has not felt welcoming to me for years. Too corporate, bad algorithms, a stream of discourse I don't care to contribute to. I just don't post publicly. But the XOXO Slack has clicked as the place to share myself with more than my IRL friends. It is ``the internet'' to me in a lot of ways. Like choosing to move to a small town where you end up knowing most folks, it feels comfortable. And this one is filled with artists and cool people. It feels like home, and I'll be really sad to see it go.
\end{itemize}

\section{People}\label{people}

Responses to the impact also talked about the people that folks had met and how important those people have become. From friendships to a deeper understanding of diverse perspectives and opinions.

\begin{itemize}
\tightlist
\item
  Just changed my life for the better. I met and connected with amazing people. I've never had a safe space like this before. It has inspired every initiative I've started since; the warmth of the community has led me to have the same attitude towards other people I encounter in the world. I've never felt so appreciated and welcomed for who I am as a person, and so unafraid to be myself. I appreciate every single person in this community and have made incredible connections with people doing inspiring things. I'm tearing up as I write this. I don't have enough words to express how I feel, and I don't think I'm doing it all justice.
\item
  Keeps me connected to so many people. Without this group I really do not have a lot of immediate social connections, especially on specific issues like parenting, portland politics, cooking, etc.
\item
  What a nice group of people.
\item
  It's really been tremendous. For the people I've met, the things I've learned, and even the career opportunities that came from being a part of a small network like this. I don't really do social media otherwise, so it's really been a salve, especially during the pandemic
\item
  This community has been the closest to feeling like ``my people'' of anything I have ever experienced. People here have been consistently welcoming and friendly. I learn about things I otherwise would have missed on an almost-daily basis. I feel honored to have this kind of access to so many amazing folks. XOXO and this community have encouraged me to be less shy about sharing things I do and make.
\item
  Just changed my life for the better. I met and connected with amazing people. I've never had a safe space like this before. It has inspired every initiative I've started since; the warmth of the community has led me to have the same attitude towards other people I encounter in the world. I've never felt so appreciated and welcomed for who I am as a person, and so unafraid to be myself. I appreciate every single person in this community and have made incredible connections with people doing inspiring things. I'm tearing up as I write this. I don't have enough words to express how I feel, and I don't think I'm doing it all justice.
\item
  I've had some professional opportunities thanks to some relationships I've made through XOXO, but those are really just the layer on top of the great personal relationships I've made through the many channels that I participate in.
\item
  The XOXO Slack has been a lifeline for me since 2018. It is a safe and caring place I can go for inspiration and information and assistance and a laugh and new music and job postings. I enjoy supporting others in the community. Being in this community gives me confidence in what I'm doing professionally. I don't feel alone. I know that people care and that there are other people like me. I am so inspired and moved by this community and the people I've met here.
\item
  Made long-term connections I could never keep up otherwise with folks across the planet
\end{itemize}

\subsection{Friendships}\label{friendships}

People spoke about the friendships in their lives because of the Slack.

\begin{itemize}
\tightlist
\item
  This community has changed my life. I've made lasting friendships and received career altering advice and assistance from the generosity and intelligence of the people assembled.
\item
  XOXO Slack and XOXO itself has been the highlight of the last decade for me. I've met some amazing friends, had a chance to catch up with my remote coworkers and see some amazing talks.
\item
  This community has been the closest to feeling like ``my people'' of anything I have ever experienced. People here have been consistently welcoming and friendly. I learn about things I otherwise would have missed on an almost-daily basis. I feel honored to have this kind of access to so many amazing folks. XOXO and this community have encouraged me to be less shy about sharing things I do and make.
\item
  I've had some professional opportunities thanks to some relationships I've made through XOXO, but those are really just the layer on top of the great personal relationships I've made through the many channels that I participate in.
\item
  I've made many new friends using a communication method that works best for me (online text), as I'm pretty socially anxious. This community has helped me become more authentic and be comfortable with who I am. I have also learned a lot from you all.
\item
  I've met so many cool people in the short time I've been in this Slack, and I was able to connect more deeply with folks I met at my first (and last) XOXO.
\item
  friendships, amazing memories, and an answer to the most esoteric questions imaginable
\item
  I've made some really wonderful lifelong friends here, and it's helped me figure out my healthiest format for a relationship with the Internet (without the risks of the actual Internet)
\end{itemize}

\subsection{Diversity}\label{diversity}

\begin{itemize}
\tightlist
\item
  This Slack is by far the most diverse community in which I participate. I grew up in tech, and most tech cohort communities are tediously mono-cultural to the point where I feel alienated from them even if I'm squarely within their demographics. Such communities also tend to have interests limited to those cohorts, and there's more to life than computers.
\item
  First and foremost, I have gained a diversity of thought. I was already an empathetic person, but it helped me understand other people's point of view better. It inspired me to not give up on humanity because there are so many cool, interesting people out there doing cool and interesting things. Many of us are perpetually online and we all have a shared understanding of THE INTERNET that my real-life friends may not have as deep of knowledge of, so I feel a sense of connection that I would crave from social media but actually achieve with XOXO.
\end{itemize}

\subsection{New to the Slack}\label{new-to-the-slack}

Folks who were newer to the Slack also talked about the impact the community had on them.

\begin{itemize}
\tightlist
\item
  ``I'm relatively new to this slack - this last conference was my first. Despite that, it has impacted me a lot. I used to get most of my community on Twitter, but after That Person took it over, I left (cold turkey, as soon as he walked in the door). That was hard, but I knew it was going to slide into a cesspool, and I decided I'd rather leave remembering how good it was than witness its death.
\item
  XOXO 2024 was my first and only XOXO, so the slack has been and continues to be a huge aspect of my connection to the community and the only non public forum I have to growing it (I am also on xoxo.zone)
\end{itemize}

\section{Resources}\label{resources-1}

Some folks talked about very resources and tangible outcomes from the slack

\begin{itemize}
\tightlist
\item
  Thank you to everyone who has been a part of this special corner of the internet. I've learned and grown thanks to this community. And, since I trust the folks here, I've found recommendations for everything from medical professionals to professional services over the years - not to mention the support and camaraderie during some hard times.
\item
  The XOXO Slack has been an invaluable resource. From everything with dealing with a global pandemic, to finding a person who can design my bathroom, to discussing video games, and more.
\item
  XOXO Slack and XOXO itself has been the highlight of the last decade for me. I've met some amazing friends, had a chance to catch up with my remote coworkers and see some amazing talks.
\item
  It was my first slack - it's where I hear about beautiful things, and stay connected to important parts of the indie web creator culture
\end{itemize}

\section{Personal Growth}\label{personal-growth}

Folks also talked about their own personal growth because of being in the community

\begin{itemize}
\tightlist
\item
  It brought me to introspect to the point that I now know what I want out of my life and I'm encouraged to pursue it.
\item
  Just changed my life for the better. I met and connected with amazing people. I've never had a safe space like this before. It has inspired every initiative I've started since; the warmth of the community has led me to have the same attitude towards other people I encounter in the world. I've never felt so appreciated and welcomed for who I am as a person, and so unafraid to be myself. I appreciate every single person in this community and have made incredible connections with people doing inspiring things. I'm tearing up as I write this. I don't have enough words to express how I feel, and I don't think I'm doing it all justice.
\item
  XOXO Slack was the setting to my third coming-of-age movie of my life.
\item
  I've gotten way better at online communications. Seeing how kind, emotionally mature, and patient people have been, and seeing moderators sharing themselves and being human, I think allows me to be more human and kind and patient myself. I also learned it's okay to vent in a different place, rather than spreading agnst.
\end{itemize}

\section{Conversations}\label{conversations}

Some folks talked about the conversations that folks have in the Slack.

\begin{itemize}
\tightlist
\item
  I joined with the 2019 XOXO and I ended up being so incredibly grateful for this slack community going into the pandemic. It helped give me hope that there were other likeminded folks out there at a time that I went through an existential tech career crisis/burnout. I wish it could just keep on staying the same way it is because I really fear that it will lose a lot of what has made it special in any kind of migration process.
\item
  I literally use the phrase ``the social Slack I'm in'' ALL THE TIME when I'm talking to friends or my spouse when I talk about things we've done, emoji we use, or threads that have gotten super wild (like the tunnel lady, or the debate stuff)
\item
  XOXO is the only space I'm in right now having important, honest, critical discussions of technology and how it affects our lives/humanity. I'm so grateful to have some role models advocating for better applications of technology and thoughtful decision making about what role technology should play in our lives
\end{itemize}

\subsection{Specific channels}\label{specific-channels}

\begin{itemize}
\tightlist
\item
  As a younger adult without many older adults in my life, I REALLY appreciate the home improvement and gardening channels as a new home owner. I love the regional channels for staying up to date with things at home and places I go. As someone less tech-y I love seeing tech updates from people I agree with politically. I also love the political updates as well! It's generally a very cathartic space in an incredibly stressful world.
\item
  There are some questions and topics I've been able to ask/discuss in the XOXO Slack that I otherwise would have no idea where I'd turn to. I've really appreciated that and will miss what the slack has become. I hope we're able to rebuild that.
\item
  The XOXO Slack has been an invaluable resource. From everything with dealing with a global pandemic, to finding a person who can design my bathroom, to discussing video games, and more.
\end{itemize}

\subsection{Portland}\label{portland-1}

\begin{itemize}
\tightlist
\item
  I moved to Portland because of it! And this is so common that it won't de-anon me!! And it's given me a much healthier relationship to my work, I don't have to set myself on fire to be a ``real'' creative person. I can take an afternoon off to take a walk and eat cookies or whatever. I can have a `capitalism job' for health insurance and money and not apologize for it.
\item
  I got a job through this slack, and it is my best go-to resource on living in Portland.
\item
  I've made friends, found out about local events, hell even the contractor upstairs installing a floor for us is literally a spouse of an XOXO'er who I found through the slack.
\item
  I moved to Portland because of it! And this is so common that it won't de-anon me!! And it's given me a much healthier relationship to my work, I don't have to set myself on fire to be a ``real'' creative person. I can take an afternoon off to take a walk and eat cookies or whatever. I can have a `capitalism job' for health insurance and money and not apologize for it.
\item
  Keeps me connected to so many people. Without this group I really do not have a lot of immediate social connections, especially on specific issues like parenting, portland politics, cooking, etc.
\item
  It was a big factor in my decision to move to the Portland area.
\end{itemize}

\subsection{Creativity}\label{creativity}

Seeing other creative folks doing things was also important to some.

\begin{itemize}
\tightlist
\item
  I feel like I have found my people. I am motivated to create and act.
\item
  It's a huge source of inspiration!
\item
  It has been so nice to see other creative people be open about their doubts and anxieties. This is such a lovely group.
\end{itemize}

\section{Care During Hard Times}\label{care-during-hard-times}

One of the themes was that people found that the community cared about them in tough times. The pandemic and challenges with mental health were common. Folks also used the term ``lifeline'' to refer to the community.

\subsection{Covid}\label{covid}

Many folks talked about how important the Slack Community was for them during the pandemic.

\begin{itemize}
\tightlist
\item
  XOXO Slack has been a place that I've gone to for comfort and a feeling of community throughout the Covid pandemic and working remotely. Although I haven't engaged much, it's been very comforting to have a place where like-minded people share things they've made and things they care about so openly.
\item
  As I told my husband the news of the Slack winding down, I began to tear up remembering how strongly I felt during the beginning/height of COVID-19 that I would never have made it out with my sanity at the other end without the XOXO Slack. Knowing that there's a community here of people who can be trusted (in many ways) is no small feat. I cherished it.
\item
  Some peace, sanity, and connection especially starting during Covid.
\item
  With the pandemic, this became one of my main sources of community in all the isolation. If it weren't for Slacks and Discords I would feel even more isolated
\item
  It's really been tremendous. For the people I've met, the things I've learned, and even the career opportunities that came from being a part of a small network like this. I don't really do social media otherwise, so it's really been a salve, especially during the pandemic
\item
  Having a safe feeling place to keep up on life, the world, tech, ai, COVID, etc, has been so incredibly valuable for me. I think I would be in a very dark place today if I didn't have the XOXO community to get me through the last few years
\item
  I joined with the 2019 XOXO and I ended up being so incredibly grateful for this slack community going into the pandemic. It helped give me hope that there were other likeminded folks out there at a time that I went through an existential tech career crisis/burnout. I wish it could just keep on staying the same way it is because I really fear that it will lose a lot of what has made it special in any kind of migration process.
\item
  Xoxo slack provided community during the darkest time of the pandemic (and beyond) . I often feel isolated from creative communities which can be cliquish. Xoxo slack is the best type of inclusive community: bringing people together to learn and grow as responsible creative beings on this planet.
\end{itemize}

\subsection{Mental Health}\label{mental-health}

Some folks spoke about the impact that the community has had on their mental health and well being.

\begin{itemize}
\tightlist
\item
  I've made many new friends using a communication method that works best for me (online text), as I'm pretty socially anxious. This community has helped me become more authentic and be comfortable with who I am. I have also learned a lot from you all.
\item
  This Slack community has gotten me through times of intense isolation (early pandemic lockdown, cross-country moves, etc). It's a place I've relied on for kind, smart, and fun connection.
\item
  Led me to ADHD and ASD diagnosis. Gifted kind words in trying times. Helped me with purchasing stuff. Made me discover some of my favorite movies. Provides an out for feelings that I have nowhere else to share. Expanded my mind and universe time and again.
\item
  I feel \emph{normal} ``here'': XOXO slack has helped me appreciate my ADHD (and possible AuDHD!) diagnosis and shown me ways to manage issues without shame. It's a venue for being funny and smart in highly targeted, small audience ways, which is very satisfying. I'm learning to share my visual art with encouragement from that channel. It's the best college dorm/fraternity\textbar sorority (that I never had).
\item
  I think the mental health channels really help me feel validated in a way that I haven't found in other online communities with similar channels. And I appreciate learning and improving how I interact with other people, even though some of the time it was the hard way.
\end{itemize}

\subsection{Lifeline}\label{lifeline}

Several people used the word ``lifeline'' when talking about the Slack or spoke about how important it was for their well being.

\begin{itemize}
\tightlist
\item
  as a disabled person who doesn't get out a lot, the community has been a lifeline for so many things
\item
  I cannot count the number of times, especially in the last four years, this community has been there for me. I'm pretty sure I'm still alive because of it.
\item
  This Slack was one of my social lifelines during the first year and a half of the pandemic and continued to be a place of solace for me in the years that followed (I had joined in summer of 2019). I don't know what life would have been like without it, but it would have been harder.
\item
  The XOXO Slack is my first stop every morning, and often one of my last stops at night. I may not always post, but it feels like the best water cooler/group chat/pin board of cool things and feelings. XOXO Slack members and the festival have helped me finally get therapy, get a better sense of boundaries, open my eyes and mind to amazingly wonderful individuals, realize that I am not alone in being hopelessly excited about good things in the world, and really did a lot of heavy lifting emotionally and socially in 2016 and 2020. Thank you friends.
\item
  XOXO Slack has been a place that I've gone to for comfort and a feeling of community throughout the Covid pandemic and working remotely. Although I haven't engaged much, it's been very comforting to have a place where like-minded people share things they've made and things they care about so openly.
\item
  XOXO Slack has helped me feel less alone online and IRL, helped me process and recognize feelings and experience I didn't know were shared and common. XOXO Slack has helped me meet some wonderful people and understand how not alone I am in the world. In practical terms, people * * I've meet through XOXO are all around me - zines, postcards, letters, stickers I've gotten over the years. XOXOers have generously gifted me furniture when I moved, pointed me to internet providers and electricity, stayed in my apartment and left me wonderful snacks and gifts and overall - always always always been helpful and amazing and omnipresent in my life for the last 8 years since I first went to the festival.
\item
  The community has improved my well being, quality of life, pointed me to resources for diversity, inclusion, equity, activism, and opened my eyes to better ways of living.
\item
  Having a safe feeling place to keep up on life, the world, tech, ai, COVID, etc, has been so incredibly valuable for me. I think I would be in a very dark place today if I didn't have the XOXO community to get me through the last few years
\item
  Has, without exaggeration, saved my life three times.
\item
  it was such a lifeline during lock down, and a sanity check during rough political waves
\end{itemize}

\section{Safe Space}\label{safe-space}

We heard again that this was a safe space.

\begin{itemize}
\tightlist
\item
  It's been a safe space.
\item
  This community has been my stable source of support for the past ten years---personally and professionally. It's been a close-knit community of like-minded folx and an online space where I don't feel like I am competing for popularity or influencer status, where I can fluctuate my engagement with little FOMO, and where curiosity or humility is embraced and not shamed or ridiculed. It's been a space that fosters a spectrum of interests and topics, whether a deep dive on a tangential topic, showcasing an inspiring creative project, or just a small-talk conversation.
\item
  Having a safe feeling place to keep up on life, the world, tech, ai, COVID, etc, has been so incredibly valuable for me. I think I would be in a very dark place today if I didn't have the XOXO community to get me through the last few years
\item
  Just changed my life for the better. I met and connected with amazing people. I've never had a safe space like this before. It has inspired every initiative I've started since; the warmth of the community has led me to have the same attitude towards other people I encounter in the world. I've never felt so appreciated and welcomed for who I am as a person, and so unafraid to be myself. I appreciate every single person in this community and have made incredible connections with people doing inspiring things. I'm tearing up as I write this. I don't have enough words to express how I feel, and I don't think I'm doing it all justice.
\item
  The internet is terrifying and I never thought I'd find a place where I felt safe enough to talk with strangers.
\item
  Thank you to everyone who has been a part of this special corner of the internet. I've learned and grown thanks to this community. And, since I trust the folks here, I've found recommendations for everything from medical professionals to professional services over the years - not to mention the support and camaraderie during some hard times.
\item
  I genuinely believe this Slack saved me through the pandemic from dealing with isolation and lost connection. I was on a medical leave from my job shortly before things shut down and relied on the Slack for daily interaction besides my spouse as a lot of other relationships fluctuated. Even just observing or reacting with emoji built connection for me that led to being more conversational, more mindful, and more confident to speak in channels.
\end{itemize}

\section{General Gratitude}\label{general-gratitude}

Finally, some folks just wanted to say thank you.

\begin{itemize}
\tightlist
\item
  Words are hard to find. SO grateful for all of it. What a special thing to have been a part of!
\item
  More ways than I can describe!!
\item
  I love this place, not sure how else to put it.
\item
  Because of that I actually feel more prepared for this evolution/change/dissolution. It has been wonderful to be a part of it, and I'm not afraid. I know I'll find my people again.
\end{itemize}

  \bibliography{book.bib,packages.bib}

\end{document}
